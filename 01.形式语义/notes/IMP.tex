\documentclass[a4paper, 12pt]{article}

\usepackage{url}
\usepackage{mathtools}

\begin{document}
    \title{IMP}
    \author{iohex}
\section{IMP}
    \emph{IMP} is a small lanuage of while programs, which called "imperative" lanuage.
    In the \emph{programming paradigms}, \emph{imperative lanuage} means program execution involves carrying out series of explicit commands to change state.

    \subsection*{syntactic sets}
    Firstly, we give the syntactic sets associated with IMP:

        \begin{itemize}
            \item numbers \textbf{N:} the set of signed decimal numerals.
            \item truth value \textbf{T}
            \item location \textbf{Loc:} non-empty strings of letters or such strings follwed by digits.
            \item arithmetic expressions \textbf{Aexp}
            \item boolean expressions \textbf{Bexp}
            \item commands \textbf{Com}
        \end{itemize}
    
    We define the \emph{formation rules} for \textbf{Aexp} by:

    \[
        a \Coloneqq n | X | a_0+a_1|a_0-a_1|a_0 \times a_1.
    \]

    %
    The symbol "::=" should be read as "can be" (p.s. BNF isn't it?)

    %
    And for \textbf{Bexp}:
    \[
        b \Coloneqq \textbf{true} | \textbf{false} | a\_0 = a\_1 | a\_0 \leq a\_1 | \neg b | b_0 \wedge b_1 | b_0 \lor b_1
    \]



\end{document}